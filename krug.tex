\documentclass{beamer}
\title{Here is my Title}
\author{Christina Huggins}
\date{July 15th, 2005}

\begin{document}

\begin{frame}
\frametitle{Optional Title for My Slide}
Here is one slide.
\end{frame}

\begin{frame}
Here is another slide.
\end{frame}

\begin{frame}
\begin{columns}[c] % the "c" option specifies center vertical alignment
\column{.5\textwidth} % column designated by a command
Contents of the first column
\column{.5\textwidth}
Contents split \\ into two lines
\end{columns}
\end{frame}

\begin{frame}
\begin{columns}[t] % contents are top vertically aligned
\begin{column}{5cm} % each column can also be its own environment
Contents of first column \\ split into two lines
\end{column}
\begin{column}[T]{5cm} % alternative top-align that's better for graphics
graphics %\includegraphics[height=3cm]{graphic.png}
\end{column}
\end{columns}
\end{frame} 

\begin{frame}
\begin{block}{Block Heading}
Enlosing text in the ``block'' environment creates a distinct, headed block of text.
\end{block}
\begin{block}{Second Block Heading}
This lets you visually distinguish parts of your slide easily.
\end{block}
\end{frame}

\begin{frame}
Since I may want to focus on one item at a time in my presentation,
\begin{itemize}
\item I want to reveal only the first item on my list initially,
\pause
\item then the second item,
\pause
\item then the third,
\pause
\item and so on...
\end{itemize}
\end{frame}

\end{document} 